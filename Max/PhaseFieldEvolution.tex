\documentclass[10pt,onecolumn]{article}

%% PREAMBLE
\usepackage{amsmath, amssymb}
\usepackage{graphics,graphicx,wrapfig}
\usepackage[colorlinks=true, linkcolor=red, citecolor=blue]{hyperref}

\usepackage{mathptmx}

\usepackage[a4paper, total={7in, 9.5in}]{geometry}
\setlength{\parindent}{0pt}
\setlength{\parskip}{0.5em}

\usepackage{sectsty}
\sectionfont{\fontsize{14}{14}\selectfont}

\pagenumbering{gobble}

\newcommand{\bs}[1]{\ensuremath{\boldsymbol{#1}}}
\newcommand{\e}{\bs{\epsilon}}
\newcommand{\s}{\bs{\sigma}}
\newcommand{\C}{\mathcal{C}}
\newcommand{\eS}{\mathcal{S}}

%%%%%%%%%%%%%%%%%%%%%%%%%%%%%%%%%%%%%%%%%%%%%%%%%%%%%%%%%%%%
%%%%%%%%%%%%%%%%%%%%%%%%%%%%%%%%%%%%%%%%%%%%%%%%%%%%%%%%%%%%
\title{Phase field evolution equation}
\author{}
\date{}

\begin{document}
\maketitle

The evolution equation for the phase field parameter $d$ is given by Eqn.(41) in \cite{miehe2010} as
%
\begin{equation}
\label{miehe41}
\frac{g_c}{l}\left[d-l^2\Delta d\right]=-g^\prime(d)\Psi_0,
\end{equation}
%
where $g(d)$ is the strain energy degradation function, $g_c$ and $l$ are constants, and $\Delta$ denotes the Laplacian operator. We rearrange Eqn.\eqref{miehe41} and get
%
\begin{equation}
\label{goveq1}
\Delta d-\frac{d}{l^2}=\frac{g^\prime(d)\Psi_0}{g_c l}.
\end{equation}
%
The ``reference'' strain energy density, $\Psi_0$, is defined in Eqn.(19) from \cite{miehe2010} as 
%
\begin{equation}
\label{strainenergy}
\Psi_0=\frac{1}{2} \e:\C:\e,
\end{equation}
%
where  $\e$ is the total infinitesimal strain tensor and $\C$ is the elastic stiffness tensor. Per Eqn.(23) in \cite{miehe2010}, the Cauchy stress $\s=g(d) \C:\e$ but by definition $\s=\C:\e_e$, where $\e_e$ is the elastic strain. Therefore,
%
\[\e_e=g(d)\e.\]
%
Replacing $\e$ with $\e_e$ in Eqn.\eqref{strainenergy} we get
%
\begin{equation}
\label{strainenergy2}
\Psi_0=\frac{1}{2}\frac{1}{g(d)^2}\e_e:\C:\e_e.
\end{equation}
%
Note that the strain energy density $\Psi=g(d) \Psi_0$ is \emph{not} the typical definition of the strain energy density and has an additional multiplicative factor of $1/g(d)$ in Miehe's formulation. If we define an elastic compliance tensor as $\eS=\C^{-1}$, then $\e_e=\eS:\s$ and
%
\begin{equation}
\label{strainenergy3}
\Psi_0=\frac{1}{2}\frac{1}{g(d)^2}\s:\eS:\s.
\end{equation}
%
Substituting this into Eqn.\eqref{goveq1}, we get
%
\begin{equation}
\label{goveq2}
\Delta d-\frac{d}{l^2}=\frac{1}{2g_c l}\frac{g^\prime(d)}{g(d)^2} \s:\eS:\s
\end{equation}

%%%%%%%%%%%%%%%%%%%%%%%%%%%%%%%%%%%%%%%%%%%%%%%%%%%%%%%%%%%%
%%%%%%%%%%%%%%%%%%%%%%%%%%%%%%%%%%%%%%%%%%%%%%%%%%%%%%%%%%%%
\bibliographystyle{unsrt}
\bibliography{bib1}

\end{document}
