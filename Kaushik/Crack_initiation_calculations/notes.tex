\documentclass[12pt,letterpaper]{article}

\usepackage{fancyhdr,lastpage}

\renewcommand{\headrulewidth}{0.4pt}\renewcommand{\footrulewidth}{0.4pt}
\lhead{Hi}\rhead{Bye}
\makeatletter
\newcommand{\resetHeadWidth}{\fancy@setoffs}
\makeatother

\pagestyle{fancy}
\fancyhead{}                    
\fancyfoot{}                    

\renewcommand{\headrulewidth}{0.5pt}
\renewcommand{\footrulewidth}{0pt}

 
\cfoot{\thepage}

\usepackage[paperwidth=9.5in, 
left=1in,
right=2in,
top=1in,
bottom=1in,
paperheight=11in, 
%margin=1in, 
footskip=.75in]{geometry}
\resetHeadWidth

\setcounter{tocdepth}{-1}
\renewcommand{\contentsname}{}

%%%%%%%%%%%%%%%%%%%%%%%%%%%%%%%%%%%%                                   
\newcommand{\required}[1]{\part*{\hfil #1\hfil}}                    %%
\renewcommand{\refname}{}                   %%
\bibliographystyle{unsrt}                                              %%


\usepackage{mathptmx}
\usepackage{floatflt}
\usepackage[svgnames, table]{xcolor}
\usepackage{array}
\usepackage{environ}
%\usepackage{tikz}
\usepackage{caption}
\usepackage{verbatim}
\usepackage{amsmath, amssymb}
%\usepackage{mathtools}
\usepackage{graphics,graphicx,wrapfig}
\usepackage[bf,small,compact]{titlesec} % Allows customization of titles
\usepackage[colorlinks=false]{hyperref}
\usepackage{showlabels}
%\usepackage{labels}
%\usepackage{todonotes}
\usepackage[font=small]{caption}
\usepackage[sort&compress]{natbib}
\urlstyle{same}
\usepackage[hang]{footmisc}
%\usepackage{enumerate}
\usepackage{enumitem}
%\usepackage{epigraph}
%\usepackage{epstopdf}
\usepackage{wrapfig}
\usepackage{dashrule}
\usepackage{pdfpages}

\setlength{\bibsep}{0.0pt}

% User defined commands
\newcommand{\unit}[1]{\ensuremath{\, \mathrm{#1}}}
\newcommand{\bs}[1]{\ensuremath{\boldsymbol{#1}}}
\newcommand{\mc}[1]{\ensuremath{\mathcal{#1}}}
\newcommand{\norm}[1]{\ensuremath \lVer #1 \rVert}
\newcommand{\rhat}{\hat{\rho}}
\newcommand{\figwidth}{2.2in}
\newcommand{\tl}{\tilde}


%Numbered environment
\newcounter{commenthk}[section]
\newenvironment{commenthk}[1][]{\refstepcounter{commenthk}\par\medskip
   \noindent \textbf{\thecommenthk. #1} \rmfamily}{\medskip}


\setlength\marginparwidth{1.5in}
\renewcommand{\comment}[2]{{\color{#1} $\blacksquare$ \footnote{\noindent \color{#1}#2} }}
\newcommand{\commenti}[2]{{\color{#1} $\blacksquare$  \color{#1}#2} }
\newcommand{\commentm}[2]{{ \color{#1} \marginpar{\color{#1} 
\scriptsize 
\begin{commenthk}
#2
\end{commenthk}
}
$^\thecommenthk$
 }}


\titleformat{\section}
  {\normalfont \large \scshape }{\thesection}{1em}{}



\titleformat{\subsection}
  {\normalfont \normalsize  \scshape \bfseries}{\thesubsection}{1em}{}


\titleformat{\subsubsection}
  {\normalfont \normalsize \scshape \bfseries}{\thesubsubsection}{1em}{}


%\titleformat{\paragraph}
%  {\normalfont  \scshape}{\theparagraph}{}{}



\usepackage{enumitem}
\setlistdepth{9}

\newlist{myEnumerate}{enumerate}{9}
\setlist[myEnumerate,1]{label=(\arabic*)}
\setlist[myEnumerate,2]{label=(\Roman*)}
\setlist[myEnumerate,3]{label=(\Alph*)}
\setlist[myEnumerate,4]{label=(\roman*)}
\setlist[myEnumerate,5]{label=(\alph*)}
\setlist[myEnumerate,6]{label=(\arabic*)}
\setlist[myEnumerate,7]{label=(\Roman*)}
\setlist[myEnumerate,8]{label=(\Alph*)}
\setlist[myEnumerate,9]{label=(\roman*)}


\let\Pi\varPi


\begin{document}
The finite element implementation of regularized variational fracture model is benchmarked for a single edge notch tension geometry subjected to far-field loading $\sigma^{\infty}$ as shown in figure.~\ref{fig1}(A). For a prescribed value of energy release rate $G_c$, the critical stress $\sigma_{cr}^{\infty}$ at which crack initiates is given by,
\begin{align*}
\sigma_{cr}^{\infty} = \sqrt{\frac{E^* G_c}{2 b \tan{\left(\frac{\pi a}{2 b} \right)} }} C\left(\frac{a}{b}\right)^{-1},
\end{align*} 
where $E^*$ is the plane strain Youngs modulus, $a$ is the length of the notch, $b$ is the width of the geometry and $C\left(\frac{a}{b}\right)$ is the compliance. The above equation is non-dimensionalized by introducing non-dimensional variables $\Pi_\mathit{1}= G_c/E^* b$ and $\Pi_\mathit{2} = a/b$ given as,
\begin{align}\label{sig_cr}
\frac{\sigma_{cr}^{\infty}}{E^*} = \sqrt{\frac{\Pi_\mathit{1}}{2  \tan{\left(\frac{\pi \Pi_\mathit{2}}{2} \right)} }} C\left(\Pi_\mathit{2}\right)^{-1}.
\end{align}

Numerical calculations are performed for the geometry shown in figure.~\ref{fig1}(A) by developing a finite element mesh with a mesh size of $h = 0.01$ close to the notch tip. Crack initiation is characterized when the value of the damage parameter $\phi$ reaches $0.99$ at the first node in front of the notch tip. The numerical results for $\sigma_{cr}^{\infty}/E^*$ is plotted against $\Pi_\mathit{2}$ for different values of $\Pi_\mathit{1}$ and are compared with the analytical solution given in eq.~\ref{sig_cr}. The solid lines correspond to the analytical solution and markers to the numerically obtained data. In figure.~\ref{fig1}(B) $\sigma_{cr}^{\infty}/E^*$ (left hand side of eq.~\ref{sig_cr}) is plotted against $\left(\Pi_\mathit{1}/2 \tan{\left(\pi \Pi_\mathit{2}/2 \right)}\right)^{1/2} C \left(\Pi_\mathit{2} \right)^{-1}$ (right hand side of eq.~\ref{sig_cr}) and the identical numerical data from figure.~\ref{fig1}(A) are compared against the reference line of slope 1. The results in both the plots compare accurately within an error of $5\%$. From these figures we can conclude that the implementation is correct and accurate to numerical precision.


\begin{figure}[b!]
	%\begin{minipage}{0.8\textwidth}
	%\begin{flushright}
	\center
	\includegraphics[width=1.0\textwidth]{./Figures/combined_plot.pdf}
	%\end{flushright}
	%\end{minipage}
	%\begin{minipage}{0.2\textwidth}
	%\begin{flushleft}
	\vspace{-8pt}
	\caption{
		{\footnotesize (A)In this figure the results of crack initiation for a single edge notch geometry of notch length $a$ and width $b$, subjected to far-field stress $\sigma^\infty$ is shown. The crack is assumed to have initiated when the value of $\phi$ at the first node in front of the notch tip reaches $0.99$ and the far-field stress at that instant is $\sigma_{cr}^{\infty}$. The non-dimensionalized stress ($\sigma_{cr}^{\infty}/E^*$) is plotted against $\Pi_\mathit{2}$. The solid lines correspond to the analytical solution given in eq.~\ref{sig_cr} and markers to the numerical data points obtained for various values of $\Pi_\mathit{1}$ and $\Pi_\mathit{2}$. (B) Identical data points from fig.~\ref{fig1}(A) are now plotted against the right hand side of the analytical formula given in eq.~\ref{fig1} and the deviation from the reference line is less than $5\%$. 
		}
		\label{fig1}
	}
	
	%\end{flushleft}
	%\end{minipage}
	\vspace{-10pt}
\end{figure}
\end{document}
